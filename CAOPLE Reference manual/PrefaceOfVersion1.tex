
\chapter*{Preface of Version 1}
\addcontentsline{toc}{chapter}{Preface of Version 1} 

CAOPLE stands for Caste-centric Agent-Oriented Programming Language
and Environment. It is an experimental programming language developed 
by the the Applied Formal Methods (AFM) Research Group \footnote{The reseach 
group has evloved into a research group called Cloud Computing and Cybersecurity 
(CCC) in March 2020} at the Oxford 
Brookes University, Oxford, England. It is a part of the research agenda of  
AFM on agent-oriented software development methodology for disciplined 
software engineering of internet-based computing. 

Agent-orientation is a novel programming paradigm proposed as an 
alternative to object-orientation. In this paradigm,
a program during execution time consists of interacting agents rather
than interacting objects. While agent-orientation inherits a number 
features from object-orientation, there are several fundamental 
differences between them. 

One of the key distinction is the mode of interaction.
An object interacts with another object by sending a message to it.
That message is guaranteed to cause the object to execute the corresponding
method. The interaction mechanism is therefore a control mechanism. In contrast, an
agent interacts with another agent by generating events and processing events. 
For example, an agent may requesting a service by generating an event, which 
may or may not be observed by the agent that provides the service. Even if the 
service provider agent observed the event, it may or may not take the action as 
requested in response. For example, it may reject the service request due to 
the request is considered as unauthorised, or simply because the workload on 
the server is too higher. In other words, the interaction mechanism is a cooperation 
mechanism. 

Request events may be thought to be analogous to method calls, but they are
asynchronous, whereas method calls are synchronous. This
paradigm is particularly suited to distributed computing, where messages
may not reach their targets. It also suited to parallel computing, where the calling
object should not suspend while waiting for the callee to finish. In both cases, a 
message is often to be broadcasted to many receivers as an action can be 
observed by many other agents rather than as in object oriented where 
a method call only targets one object. 
Both of forms of computing have been the subject of intense
study over the past few decades, but have now become far more important
with the advent of web services and multi-core computing.

In CAOPLE, each agent is an instance of a caste, which can therefore
be thought of as a template for creating agents; the relationship
is analogous to that between objects and classes. The major difference
between classes and castes though is that an agent can change its
caste membership during its life-time whereas an object cannot. 
Many other agent-oriented
programming languages do not have caste, or anything similar, as a
classification mechanism so the concept of caste is an important contribution
by providing a means of modularity. 

Another similarity to object oriented programming languages, CAOPLE is 
an imperative programming language while most other agent-oriented 
programming languages are declarative or hybrid (i.e. a combination of 
imperative and declarative facilities). 

Finally, CAOPLE is a pure agent-oriented programming language, in which 
a program only contains agents rather than a mixture of agents and 
objects as many other agent-oriented programming languages. 

This manual intends to give a brief but comprehensive definition of the language 
rather than an introduction to agent-oriented programming. 
