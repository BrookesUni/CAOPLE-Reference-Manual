
\chapter*{Preface of Version 2}
\addcontentsline{toc}{chapter}{Preface of Version 2}

Since the writing of CAOPLE manual version 1 in the summer of 2014, 
the AOP research team has made substantial progress in the development of the 
language. The virtual machine CAVM has been redeveloped written in Java and 
a compiler has also been re-written with JavaCC and Java. An integrated development 
environment called CIDE has also been developed. As a consequence, the language has 
changed significantly in its syntax. These changes are now reflected in this version of the
language manual. 

It is worth noting that during this period of time, the microservices architecture of service-oriented computing 
has become a hot topic of research and widely accepted by industry for the development of service-oriented systems. 
Microservices is a software architecture style in which a large complex software application is decomposed into many services, each of small size (hence the word \emph{micro} in its name). These services can be independently deployed to a cluster of computers and duplicated to achieve scalability, load balance and system-performance optimisation. Each micro-service runs in its own process and interacts with other services through lightweight communication mechanisms. In our research paper titled ``\emph{CAOPLE: A Programming Language for Microservices SaaS}'' \footnote{Chengzhi Xu, Hong Zhu, Ian Bayley, David Lightfoot, Mark Green and Peter Marshall, CAOPLE: A Programming Language for Microservices SaaS, in Proceedings of \emph{the 10th IEEE International Conference on Service Oriented System Engineering (SOSE 2016)}, 29 March - 1 April 2016, Oxford, England, UK.  IEEE Computer Society Press, 2016.}, we argued that CAOPLE is particularly suitable for programming software in the microservices architecture. Below we briefly summarise the argument. 

In the literature of service-oriented computing, the word ``service'', and similarly the word ``microservice'', has two meanings. First, a service is the functionality provided by a computer system and delivered to the users; for example, as Singh and Huhns stated \footnote{Singh, P. M., and Huhns, N. M.,  \emph{Service-Oriented Computing: Semantics, Processes, Agents.} Wiley, 2005.}. Second, the word service also refers to the computational entities that provide the services in the first sense. Here, we separate these two concepts by using the word service only to refer to the functionalities that a computational system provides, while the computational entities that provide such functionality are called ``agents''. In other words, agents are service providers. Of course, an agent may well need services from other agents. Thus, they can be, and often are, service requesters, too. 

In the context of the microservices architecture, the word microservice also bears two further meanings: first, a set of microservices can be identical copies of a ``service'', where each copy is a runtime computational entity. Second, given the existence of multiple copies running in a system, a microservice is also referred to as a template from which instances can be generated and deployed to different servers. In our conceptual model, the notion of \emph{caste} captures perfectly the meaning of a template from which runtime instance are instantiated. The implementation of CAOPLE enables that instances of a caste run not just on the same machine, but, as a norm to be run over a computer network. 

Therefore, the notion of agents and their classifier caste provides a perfect conceptual model of microservices. 

Microservices architecture are not easy to develop, deploy and operate because the complexity of the system increases tremendously when it is decomposed into a large number of micro scale services. With the wide acceptance of microservices architecture in industry for cloud computing, container technologies, such as docker, has been developed to support the deployment of software in microservice architecture over a large scale computer network and monitoring the operation of the system. How to develop such systems is still an open problem. CAOPLE addresses this by significantly reducing the complexity of programming distributed systems with a higher level of abstraction and well-defined conceptual model of microservices architecture so that a system in microservices architecture can be represented in a nice model in a natural metaphor. Moreover, its network transparency make the deployment, testing and debugging distributed system less difficult than any existing programming languages. Therefore, CAOPLE as a new programming language providers an alternative approach to the container technology currently used in the industry. In our another research paper, which is titled ``\emph{CIDE: An Integrated Development Environment for Microservices}'' \footnote{Desheng Liu,  Hong Zhu, Chengzhi Xu,  Ian Bayley, David Lightfoot, Mark Green and Peter Marshall, CIDE: An Integrated Development Environment for Microservices, in Proceedings of \emph{2016 IEEE International Conference on Services Computing (SCC 2016)},  27 June - 2 July 2016, San Francisco, CA, USA. IEEE Computer Society Press, 2016.},  we have proposed a new generation of integrated development environment that integrates IDE with new functionalities that are currently offered by deployment and operation tools to enable the so-call \emph{shift-to-the-left} principle of DevOps software development philosophy. A prototype of an IDE for CAOPLE programming language called CIDE has been developed. Our own experiences with writing example codes show that developing distributed and parallel applications in CAOPLE with CIDE can be much easier, faster, and the code is more readable and testable. 

Finally, it is worth noting that CAOPLE is not just for microservices. It represents a new paradigm of programming that also suitable for other types of parallel computing and distributed computing. 
